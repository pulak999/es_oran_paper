\section{Related Work}
\label{sec:related}

Will fill in after seeing how much space is left. \\

\begin{comment}
We briefly discuss related work that is not covered in~\cref{sec:background}.

\para{Implementing cryptographic algorithms on programm-able switches}
There have been several efforts on implementing cryptographic primitives on programmable switches like encryption schemes (e.g., P4-AES~\cite{2020-SPIN-P4AES}, ChaCha~\cite{2022-EuroP4-ChaCha}), and secure keyed hash functions (SipHash~\cite{2021-SPIN-HalfSipHash}).
However, existing schemes cannot provide the three necessary security properties to ensure end-to-end secure communication, and thus \sysname is orthogonal with the aforementioned.
This also makes \aead the first cryptographic primitive in the data plane to support secure communication channels.
% While one can compose P4-AES and SipHash to achieve similar characteristics of an authenticated encryption scheme, it is not proven to be secure~\cite{rogaway2002authenticated}.
% Also, given the composite of two algorithms, more hardware resources are required to be dedicated.
While one can compose P4-AES and SipHash to construct an authenticated encryption scheme, it is not proven to be secure~\cite{rogaway2002authenticated} and it requires more dedicated hardware resources.
Further, the P4-AES approach is not scalable, as the number of sessions is strictly limited by the memory available to maintain the per-key precomputed lookup tables.
In contrast, \aead requires little memory (see~\cref{sec:implementation}) to maintain the constants (i.e., 12 $\times$ 16-bit numbers) used for the {\pround}s.
% given its lightweight property.
Given the low resource requirement of \ascon, it can be integrated into and co-exist with existing data plane programs to secure the in-band communication channels.
\end{comment}