\begin{abstract}

    Mobile communication technologies, in their quest to deliver the highest possible data rates over the air interface, have nearly touched the Shannon limit \textcolor{blue}{[CITE]}. 
    %\begin{comment}[\textcolor{blue}{CITE}]\end{comment}. 
    This has been made possible through the implementation of Multi-user MIMO, beamforming, and the utilization of large bandwidths in the C-band and millimeter-wave spectrum \textcolor{blue}{[CITE]}.
    Despite the advancements in techniques to enhance service capabilities, the power usage by the radio and compute components at the base stations (known as eNodeB in LTE and gNodeB in NR) frequently constitutes a significant part of the operational costs (OPEX) for operators, a factor that is commonly disregarded.
    This paper delves into one of the most straightforward strategies for energy conservation in cellular networks: deactivating under-utilized cells.
    Although this technique has been extensively researched, we introduce a \textcolor{red}{novel} validation framework for evaluating its effectiveness and ensuring the Quality of Service (QoS) is maintained for end users.
    Our simulations, conducted over a rApp setup, demonstrate the effectiveness of the proposed technique, resulting in a significant reduction of \textcolor{red}{[VALS]} in power consumption.
    
\end{abstract}
