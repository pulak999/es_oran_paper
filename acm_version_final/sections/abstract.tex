\begin{abstract}

Energy consumption is one of the most pressing challenges in the use of radio access networks (RANs), making it a critical area for research. Recent studies show that RANs contribute significantly to the total energy usage of mobile networks. The Open Radio Access Network (O-RAN) architecture, particularly with the use of xApps and rApps interfaced with RAN Intelligent Controllers (RIC), offers promising solutions for effective energy management in RANs. %Designed to be more flexible and scalable than traditional RANs, the O-RAN architecture enables more efficient utilization of network resources.
    
In this work, we have developed a platform-agnostic Energy-Saving (ES) rApp which will interface with any O-RAN compliant networks. The rApp comprises multiple components, including a simple LSTM model for predicting the network's short-term energy needs, and a separate decision-making entity that decides which cells to switch off to save energy. %The model is trained on historical network data and uses this data to forecast future energy trends.
\textcolor{red}{[PULAK] more infomation of our testing environment}
The obtained simulation results demonstrate the energy efficiency and performance approach of the proposed control in terms of a \textcolor{red}{[PULAK] After results.}\% reduction in power consumed as well as \textcolor{red}{an increase in coverage [PULAK] After results.}. 
This improvement is achieved without affecting the overall throughput and CQI of the connected UEs.



\end{abstract}
