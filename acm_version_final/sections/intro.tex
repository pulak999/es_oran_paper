\section{Introduction}
\label{sec:intro}

With the advancements in techniques to pack maximum data over the communication channel, the Information and Communication Technology (ICT) industry has devised technologies which help us reach the Shannon’s limit on amount of data that can be reliably transferred over a channel. 
These include efficient channel coding techniques like Low Density Parity Checks (LDPC) to spatial re-use of channel as in Multi-user MIMO coupled with beamforming to direct data to specific users or set of users. 
In order to meet the ever-increasing service demands, more spectra has been harvested for mobile communication networks, like the C-band which provides hundreds of MHz of bandwidth and millimeter wavelength spectrum which offers GHz of spectrum. 
What is often overlooked are the aspects of practical realization of these techniques which demand huge computation resources both at the radio units as well as for processing in base-band. 
Achieving energy efficiency in mobile networks would be the largest problem facing the ICT industry to meet their goal of net zero emissions by 2050. 

Within a mobile network, Radio Access Networks (RAN) have been found to be one the most significant users of a mobile network's total power supplied.
%[Green Future Networks – Sustainability Challenges and Initiatives in Mobile Networks by NGMN Alliance, December 2021, \href{https://www.ngmn.org/wp-content/uploads/210719_NGMN_GFN_Sustainability-Challenges-andInitiatives_v1.0.pdf}]. [https://arxiv.org/abs/2301.06713]]
Recent initiatives such as Open RAN (O-RAN) [] have introduced the concept of RAN Intelligent Controllers (RICs) as flexible platforms for robust RAN control, platforms which we put into use.
The disaggregation introduced in O-RAN paves way for the introduction of several network optimization techniques without disturbing  core functionality.
O-RAN control is enabled using applications called xApps (for Near-Real-Time RIC) and rApps (for Non-Real-Time RIC), with the choice of the implementation made depending on the time-frame of the control.

This paper examines one of the simplest techniques for energy savings - powering off unused nodes. 
It aims to establish formalisms for analyzing this technique and its impact on the Quality of Service (QoS) as perceived by the end users.
With the flexibility of implementation intrroduced by the O-RAN initiative, we have tested out the algorithim in a rApp compatible with any O-RAN compliant network. 
The main contributions of this paper are as follows:
\begin{itemize}
    \item Introduced a statistical metric to evaluate the effectiveness of implemented policies, offering a method to validate decisions prior to implementation.
    \item Developed a swift and efficient method for cell shutdown, eliminating the need for excessive KPI calls or extensive training time.
    \item Implemented and detailed a solution that is readily deployable across all O-RAN compliant networks.
\end{itemize}

\textcolor{red}{How is this different from all the other implementations? Is this just a simple implementation of a pre-existing idea in an O-RAN specification?}\\
Usual heuristic heavy approaches to cell shutdown/bringup:
1. toggle between MIMO and SISO --> Machine Learning-Based MIMO Enabling Techniques for Energy Optimization in Cellular Networks (IEEE) \\
2. does what we do, except using GS-STN and RL --> Deep Reinforcement Learning With Spatio-Temporal Traffic Forecasting for Data-Driven Base Station Sleep Control (ACM) \\
3. more focused on TS xApp bit --> Energy Optimization in Ultra-Dense Radio Access Networks via Traffic-Aware Cell Switching (IEEE) \\
4. predict BBUs based on RRH traffic in a CRAN env. --> Traffic Prediction-Enabled Energy-Efficient Dynamic Computing Resource Allocation in CRAN Based on Deep Learning (IEEE) \\
5. similar, but QoS is guaranteed using spectral efficiency --> A New Heuristic Algorithm for Energy and Spectrum Efficient User Association in 5G Heterogeneous Networks (IEEE) \\ 
6. properly heuristic --> Energy Saving in 5G Cellular Networks using Machine Learning Based Cell Sleep Strategy (IEEE), An Efficient Energy-Saving Scheme Using Genetic Algorithm for 5G Heterogeneous Networks (IEEE) \\  

The remainder of this paper is organized into six sections, as follows.
Section 2 provides a discussion of current approaches to energy saving with the RAN stack, and how our approach has it's own merits. 
Section 3 contains an overall overview of our energy saving solution, while scxtion 4 provides an algorithm to arrive at optimal solution.
The results obtained from the rApp evaluation using the software-defined NS-3 simulator are depicted and discussed in Section 5. 
Finally, Section 6 concludes the paper and suggests future research directions.\\

\definecolor{mygrey}{RGB}{145,146,156}
\definecolor{mypurple}{RGB}{153,102,204}
\definecolor{myotherblue}{RGB}{0,191,255}
\definecolor{mygreen}{RGB}{144,238,144}
\definecolor{myyellow}{RGB}{255,215,0}
\definecolor{myorange}{RGB}{255,165,0}
\definecolor{myothergreen}{RGB}{35,79,30}
\definecolor{myred}{RGB}{255,127,80}
\definecolor{myindigo}{RGB}{100,110,230}

\begin{figure}
    \centering
    \begin{tikzpicture}[
        subox1/.style={draw, fill=myorange, minimum width=1.1cm, minimum height=0.7cm, font=\footnotesize},
        subox2/.style={draw, fill=myindigo, minimum width=0.55cm, minimum height=0.7cm, font=\footnotesize},
        subox3/.style={draw, fill=myred, minimum width=0.55cm, minimum height=0.7cm, font=\footnotesize},
        subox4/.style={draw, fill=myothergreen, minimum width=2.4cm, minimum height=0.4cm, font=\footnotesize},
        box1/.style={draw, fill=mygreen, minimum width=2.6cm, minimum height=1.5cm, font=\footnotesize},
        box2/.style={draw, fill=myyellow, minimum width=1.7cm, minimum height=1.5cm, font=\footnotesize},
        smallbox1/.style={draw, fill=mypurple, minimum width=0.85cm, minimum height=0.5cm, font=\footnotesize, rounded corners},
        smallbox2/.style={draw, fill=myotherblue, minimum width=0.85cm, minimum height=0.5cm, font=\footnotesize, rounded corners},
        smallbox3/.style={draw, fill=mygrey, minimum width=0.85cm, minimum height=0.5cm, font=\footnotesize, rounded corners},
        flatbox1/.style={draw, fill=myotherblue, minimum width=2cm, minimum height=0.25cm, font=\footnotesize},
        flatbox2/.style={draw, fill=myotherblue, minimum width=3.2cm, minimum height=0.25cm, font=\footnotesize},
        flatbox3/.style={draw, fill=mygrey, minimum width=3.2cm, minimum height=0.35cm, font=\footnotesize},
        bigbox1/.style={draw, fill=myotherblue, minimum width=0.8cm, minimum height=0.7cm, font=\footnotesize},
        bigbox2/.style={draw, fill=mygrey, minimum width=0.75cm, minimum height=0.7cm, font=\footnotesize},
        bigbox3/.style={draw, fill=mygrey, minimum width=4.45cm, minimum height=0.4cm, font=\footnotesize},
        hugebox/.style={draw, , fill=mygrey, minimum width=9cm, minimum height=0.75cm, font=\footnotesize},
        scale=0.4 % Adjust this value to fit in one or two columns
    ]
    
    % Base layer
    \node[hugebox, text=white] (ns3) {\textbf{\texttt{ns-3 (network simulator)}}};

    % Second layer
    \node[smallbox1, text=white, above left=0.2 and -1.5 of ns3] (sdnr) {\texttt{SDNR}};
    \node[flatbox1, text=white, above left=0.2 and -4.5 of ns3] (o1) {\texttt{O1-adapter}};
    \node[flatbox2, text=white, above right=0.2 and -3.2 of ns3] (e2) {\texttt{E2-adapter, USOI}};
    \node[flatbox3, text=white, above right=0.75 and -3.2 of ns3] (near-ric) {\textbf{\texttt{Near-RT RIC}}};
    
    % Third layer
    \node[bigbox1, text=white, above right=1.25 and -0.95 of ns3, align=center] (other-x) {\texttt{other} \\ \texttt{xApps}};
    \node[bigbox2, text=white, above right=1.25 and -2.27 of ns3, align=center] (kpi-x) {\textbf{\texttt{KPI-MON}} \\ \textbf{\texttt{xApp}}};
    \node[bigbox2, text=white, above right=1.25 and -3.15 of ns3, align=center] (ts-x) {\textbf{\texttt{TS}} \\ \textbf{\texttt{xApp}}};
    
    \node[bigbox3, text=white, above left=0.85 and -4.5 of ns3, align=center] (smo) {\textbf{\texttt{SMO}}};
    \node[bigbox3, text=white, above left=1.4 and -4.5 of ns3, align=center] (non-ric) {\textbf{\texttt{Non-RT RIC}}};

    % Fourth layer
    \node[box1, above left=0.15 and -2.6 of non-ric] (es) {};
    \node[box2, text=white, above left=0.15 and -4.45 of non-ric, align=center] (rapps) {\texttt{other} \\ \texttt{rApps}};
    
    % Super-Impose layer
    \node[subox1, text=white, above left=0.3 and -1.2 of non-ric, align=center] (dt) {\textbf{\texttt{DT}}};
    \node[subox2, text=white, above left=0.3 and -1.85 of non-ric, align=center] (cp) {\textbf{\texttt{CP}}};
    \node[subox3, text=white, above left=0.3 and -2.5 of non-ric, align=center] (tp) {\textbf{\texttt{TP}}};
    \node[subox4, text=white, above left=1.1 and -2.5 of non-ric, align=center] (dme) {\textbf{\texttt{DME}}};
    \node[smallbox1, text=white, above left=0.55 and -3.9 of ns3, align=center] (ves) {\texttt{VES} \\ \texttt{clctr}};
    
    % Arrows

    %\draw[-, line width=0.5mm] (sdnr.south) -- (ns3.north -| sdnr.south);
    %\draw[-, line width=0.5mm] (o1.south) -- (ns3.north -| o1.south);
    %\draw[-, line width=0.5mm] (e2.south) -- (ns3.north -| e2.south);
    %\draw[-, line width=0.5mm] (near-ric.south) -- (e2.north -| near-ric.south);
    %\draw[->, line width=0.5mm] (non-ric.south) -- (e2.north -| non-ric.south);
    
    % Text
    %\node[rotate=90, red, anchor=south, font=\tiny] at ($(es.west)+(-0.4,0)$) {Cell ON/OFF};
    \node[above=0.01 of es] {\small \textbf{\texttt{ES rApp}}};

    \end{tikzpicture}
    \caption{System Architecture Diagram}
    \label{fig:system-architecture}
\end{figure}
    

\begin{comment}
The remainder of this paper is organized into six sections, as follows. 
Section 2 provides some more background on the topic and a discussion of current approaches to energy saving with the RAN stack. 
Section 2 contains an overall overview of the rApp and it's functioning. 
Section 3 presents the architecture and the overall flow of the proposed energy saving algorithm. 
Section 4 presents the underlying rationale utilized by the proposed approach and then details the model selection and training. 
The results obtained from the rApp evaluation in the software-defined O-RAN simulation are depicted and discussed in Section 5. 
Finally, Section 6 concludes the paper and suggests future research directions.\\

%\textcolor{blue}{What is the problem you are trying to solve? Give some background to it}\\
% Intro
With the rise in Internet literacy network providers face the challenge of creating a robust infrastructure that supports a growing number of users while also accommodating increasingly complex and data-intensive applications.
This expansion, along with the advent of next-generation networks, results in larger and more intricate networks. 
As the number of network nodes increases, the energy required to maintain such dense and complex cellular networks also escalates, making their energy consumption a significant concern [\textcolor{blue}{might cite something here}].
%Therefore, it is crucial to develop energy-efficient strategies for managing and operating these cellular networks. This includes optimizing the use of network resources, implementing energy-saving technologies, and exploring renewable energy sources for network operations.

%\textcolor{blue}{why energy saving/control is difficult in normal RAN}
Radio Access Networks (RAN) have been found to be one the most significant users of a mobile network's total power supplied [Green Future Networks – Sustainability Challenges and Initiatives in Mobile Networks by NGMN Alliance, December 2021, \href{https://www.ngmn.org/wp-content/uploads/210719_NGMN_GFN_Sustainability-Challenges-andInitiatives_v1.0.pdf}]. [https://arxiv.org/abs/2301.06713]]
In today's date, a RAN comprises of several cell sites with site infrastructure equipment and base station equipment. 
Existing RAN infrastructures are often built on older technologies that are not optimized for energy efficiency, making retrofitting for energy savings complex and costly. [\textcolor{blue}{might cite something here}]
To address the challenge of energy consumption in cellular networks, especially with the widespread usage of technologies like ultra-dense networks (UDNs) and network slicing [\textcolor{blue}{might cite something here}], it is crucial to understand the current energy usage patterns. Studies have shown that 5G base stations consume approximately three times more power than their 4G counterparts, primarily due to the need for denser deployments and advanced technologies like massive MIMO antennas. [https://spectrum.ieee.org/5gs-waveform-is-a-battery-vampire]

% ORAN Intro
%\textcolor{blue}{what is the ORAN concept and how it can be used to reduce complexity of the network. how it gives freedom}
On the other hand, virtualization makes it possible for network functions and resources to be performed and allocated to different parts of the networks in a dynamic matter thus making the RAN-as-a-service rather than as dedicated hardware as obtained in the previous generations of RANs [P. K. Thiruvasagam, V. Venkataram, V. R. Ilangovan, M. Perapalla, R. Payyanur, V. Kumar et al., Open RAN: Evolution of Architecture Deployment Aspects and Future Directions, 2023.]
O-RAN's open interfaces and standardized architecture allow for advanced algorithms that dynamically allocate resources based on real-time traffic demands, thereby optimizing energy usage.

% basic intro of solution
The programmability of O-RAN facilitates the deployment of AI-driven solutions that can predict traffic patterns and adjust energy consumption proactively, further enhancing efficiency. [https://www.tatacommunications.com/blog/2022/03/14/5g-and-open-ran-how-they-are-transforming-the-telecom-industry/]
The Non-RT RIC is responsible for the energy-saving algorithm execution. The algorithm is triggered by the SMO
The non-RT RIC, in particular, is designed to handle tasks that do not require immediate response, making it ideal for applications focused on long-term optimization and strategic planning, such as Energy Efficiency.

First, the E2 Nodes are configured by the Service Management and Orchestration (SMO) to report the data necessary for energy-saving algorithms via the O1 Interface to the Collection and Control unit. Assuming that the Non-RT RIC and SMO are tightly coupled the NonRT RIC retrieves the collected data through internal SMO communication \textcolor{green}{how???}. The O-RUs are involved in this use case. The E2 Nodes need to configure them to report data through the Open RAN Fronthaul Management Plane (Open FH M-Plane) interface.\\
Before switching off/on carrier(s) and/or cell(s), the E2 Node may need to perform some preparation actions for off switching (e.g. check ongoing emergency calls and warning messages, to enable, disable, modify Carrier Aggregation and/or Dual Connectivity, to trigger HO traffic and UEs from cells/carriers to other cells or carriers, informing neighbour nodes via X2/Xn interface etc.) as well as for on switching (e.g., cell probing, informing neighbour nodes via X2/Xn interface etc.). \\

The expert reader may argue why we focus on implemeting a rApp instead of an xApp considering it's shorter timeframe of operation, and therefore control. There are several reasons. 

The key insights (denoted as “I”) resulting from our analysis can be summarized as follows:
- I 1: We confirm with a recent study an old finding that datesback to the year 2004 [16]: data-plane traffic exhibitsself-similarity properties at both individual uplink anddownlink components and as a whole. This is differentfrom control-plane mobile network traffic [29] \\
- I 2: We find that the number of Radio Resource Control (RRC )connected users follows a bi-modal distribution that in-dicates the presence of circadian cycles resulting in twoclusters of different sizes, i.e., users connected during theday and users connected during the night. \\

\end{comment}