\section{Introduction}
\label{sec:intro}

%\textcolor{blue}{What is the problem you are trying to solve? Give some background to it}\\
% Intro
With the rise in Internet literacy network providers face the challenge of creating a robust infrastructure that supports a growing number of users while also accommodating increasingly complex and data-intensive applications.
This expansion, along with the advent of next-generation networks, results in larger and more intricate networks. 
As the number of network nodes increases, the energy required to maintain such dense and complex cellular networks also escalates, making their energy consumption a significant concern [\textcolor{blue}{might cite something here}].
%Therefore, it is crucial to develop energy-efficient strategies for managing and operating these cellular networks. This includes optimizing the use of network resources, implementing energy-saving technologies, and exploring renewable energy sources for network operations.

%\textcolor{blue}{why energy saving/control is difficult in normal RAN}
Radio Access Networks (RAN) have been found to be one the most significant users of a mobile network's total power supplied [Green Future Networks – Sustainability Challenges and Initiatives in Mobile Networks by NGMN Alliance, December 2021, \href{https://www.ngmn.org/wp-content/uploads/210719_NGMN_GFN_Sustainability-Challenges-andInitiatives_v1.0.pdf}]. [https://arxiv.org/abs/2301.06713]]
In today's date, a RAN comprises of several cell sites with site infrastructure equipment and base station equipment. 
Existing RAN infrastructures are often built on older technologies that are not optimized for energy efficiency, making retrofitting for energy savings complex and costly. [\textcolor{blue}{might cite something here}]
To address the challenge of energy consumption in cellular networks, especially with the widespread usage of technologies like ultra-dense networks (UDNs) and network slicing [\textcolor{blue}{might cite something here}], it is crucial to understand the current energy usage patterns. Studies have shown that 5G base stations consume approximately three times more power than their 4G counterparts, primarily due to the need for denser deployments and advanced technologies like massive MIMO antennas. [https://spectrum.ieee.org/5gs-waveform-is-a-battery-vampire]

% ORAN Intro
%\textcolor{blue}{what is the ORAN concept and how it can be used to reduce complexity of the network. how it gives freedom}

On the other hand, virtualization makes it possible for network functions and resources to be performed and allocated to different parts of the networks in a dynamic matter thus making the RAN-as-a-service rather than as dedicated hardware as obtained in the previous generations of RANs [P. K. Thiruvasagam, V. Venkataram, V. R. Ilangovan, M. Perapalla, R. Payyanur, V. Kumar et al., Open RAN: Evolution of Architecture Deployment Aspects and Future Directions, 2023.]

Recent initiatives such as Open RAN (O-RAN) [ORAN Alliance, 2023, [online] Available: https://www.o-ran.org/.] have introduced the concept of RAN Intelligent Controllers (RICs) as a flexible platform for robust RAN control.
O-RAN control is enbaled using applications called xApps (for Near-Real-Time RIC) and rApps (for Non-Real-Time RIC), with the choice made depending on the time-frame of the control. [\textcolor{blue}{might cite something here}]
O-RAN's open interfaces and standardized architecture allow for advanced algorithms that dynamically allocate resources based on real-time traffic demands, thereby optimizing energy usage.

% basic intro of solution
The programmability of ORAN facilitates the deployment of AI-driven solutions that can predict traffic patterns and adjust energy consumption proactively, further enhancing efficiency. [https://www.tatacommunications.com/blog/2022/03/14/5g-and-open-ran-how-they-are-transforming-the-telecom-industry/]
The non-RT RIC, in particular, is designed to handle tasks that do not require immediate response, making it ideal for applications focused on long-term optimization and strategic planning.
Switching off under-loaded cells during network operation without affecting the user experience (call drops, QoS degradation, etc.) is one way to achieve RAN energy efficiency. A typical energy savings scenario is realized when capacity booster cells are deployed under the umbrella of cells providing basic coverage and the capacity booster cells are switched off to enter dormant mode when its capacity is no longer needed and reactivated on a need basis. - First, the E2 Nodes are configured by the Service Management and Orchestration (SMO) to report the data necessary for energy-saving algorithms via the O1 Interface to the Collection and Control unit. Assuming that the Non-RT RIC and SMO are tightly coupled the NonRT RIC retrieves the collected data through internal SMO communication \textcolor{green}{how???}. The O-RUs are involved in this use case. The E2 Nodes need to configure them to report data through the Open RAN Fronthaul Management Plane (Open FH M-Plane) interface.\\
The Non-RT RIC is responsible for the energy-saving algorithm execution. The algorithm is triggered by the SMO and the Non-RT RIC is responsible for the algorithm execution. The Non-RT RIC is responsible for the algorithm execution. The algorithm is triggered by the SMO and the Non-RT RIC is responsible for the algorithm execution. The Non-RT RIC is responsible for the algorithm execution. The algorithm is triggered by the SMO and the Non-RT RIC is responsible for the algorithm execution. The Non-RT RIC is responsible for the algorithm execution. The algorithm is triggered by the SMO and the Non-RT RIC is responsible for the algorithm execution. The Non-RT RIC is responsible for the algorithm execution. The algorithm is triggered by the SMO and the Non-RT RIC is responsible for the algorithm execution. The Non-RT RIC is responsible for the algorithm execution. The algorithm is triggered by the SMO and the Non-RT RIC is responsible for the algorithm execution. The Non-RT RIC is responsible for the algorithm execution. The algorithm is triggered by the SMO and the Non-RT RIC is responsible for the algorithm execution. The Non-RT RIC is responsible for the algorithm execution. The algorithm is triggered by the SMO and the Non-RT RIC is responsible for the algorithm execution. The Non-RT RIC is responsible for the algorithm execution. The algorithm is triggered by the SMO and the Non-RT RIC is responsible for the algorithm execution. The Non-RT RIC is responsible for the algorithm execution. The algorithm is triggered by the SMO and the Non-RT RIC is responsible for the algorithm execution. The Non-RT RIC is responsible for the algorithm execution. The algorithm is triggered by the SMO and the Non-RT RIC is responsible for the algorithm execution. The Non-RT RIC is responsible for the algorithm execution. The algorithm is triggered by the SMO and the Non-RT RIC is responsible for the algorithm execution. The Non-RT RIC is responsible for the algorithm execution. The algorithm is triggered by the SMO and the Non-RT RIC is responsible for the algorithm execution. The Non-RT RIC is responsible for the algorithm execution. The algorithm is triggered by the SMO and the Non-RT RIC is responsible for the algorithm execution. The Non-RT RIC is

The expert reader may argue why we focus on implemeting a rApp instead of an xApp considering it's shorter timeframe of operation, and therefore control. There are several reasons. 

Before switching off/on carrier(s) and/or cell(s), the E2 Node may need to perform some preparation actions for off switching (e.g. check ongoing emergency calls and warning messages, to enable, disable, modify Carrier Aggregation and/or Dual Connectivity, to trigger HO traffic and UEs from cells/carriers to other cells or carriers, informing neighbour nodes via X2/Xn interface etc.) as well as for on switching (e.g., cell probing, informing neighbour nodes via X2/Xn interface etc.). \\

- How is this different from all the other implementations? Is this just a simple implementation of a pre-existing idea in an O-RAN specification?\\
To the extent of our knowledge, none of the state-of-the-art approaches has tackled this issue. To fill this gap, we present in this paper a novel serial algorithm that has been designed to handle all network topologies compliant with the O-RAN architecture.
- To bypass holes and ensure query completeness, the pro-posed algorithm leverages boundary traversal.\\
- To eliminate collisions, the proposed algorithm utilizes asingle packet to visit nodes, query them, and collect theiranswers.\\
- To ensure scalability, the proposed algorithm has been de-signed to rely only on the local one-hop information avail-able at the level of each visited node (no extra informationis needed).\\
- To be robust against failures and topology changes, the pro-posed approach does not require any structure building ormaintenance. The traversal path is gradually constructed byeach visited node. \\

- What are the insights from our experiments? - Results Part \\
The key insights (denoted as “I”) resulting from our analysis can be summarized as follows:
- I 1: We confirm with a recent study an old finding that datesback to the year 2004 [16]: data-plane traffic exhibitsself-similarity properties at both individual uplink anddownlink components and as a whole. This is differentfrom control-plane mobile network traffic [29] \\
- I 2: We find that the number of Radio Resource Control (RRC )connected users follows a bi-modal distribution that in-dicates the presence of circadian cycles resulting in twoclusters of different sizes, i.e., users connected during theday and users connected during the night. \\

The remainder of this paper is organized into six sections, as follows. Section 2 provides some more background on the topic and a discussion of current approaches to energy saving with the RAN stack. Section 3 contains an overall overview of the rApp and it's functioning. Section 4 presents the architecture and the overall flow of the proposed energy saving algorithm. Section 5 presents the underlying rationale utilized by the proposed approach and then details the model selection and training. The results obtained from the rApp evaluation in the software-defined O-RAN simulation are depicted and discussed in Section 6. Finally, Section 7 concludes the paper and suggests future research directions.\\