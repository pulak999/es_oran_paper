\section{Conclusion and Future Work}
This work aimed to develop and evaluate an Energy Saving rApp for the O-RAN architecture using decision making algorithms and LSTMs for prediction. A LSTM neural network model was trained on various time-series datasets \textcolor{blue}{generated using the NIST radar (incum-bent) waveform generator}. Also, 5G NR data was capturedusing the ESC sensor node with the SDR based CBSD asthe signal source. The collected 5G NR data was mixed withthe NIST generated datasets to train the ML model for 5GNR non-incumbent detection. The model training resultsshow better performance at higher SNR values as expected.The highest prediction accuracy of 95.83\% was achieved for \textcolor{blue}{[PULAK - Edit in the end] dataset with signals in the SNR range 40-50 dB}. OTA incum-bent detection is achieved by transmitting the signals in thedataset OTA and results are presented. In the OTA resultsfor incumbent detection, the highest accuracy achieved is 85.35\%. Additionally, the results for the OTA non-incumbent(5G-NR) signal is presented. The model achieves an accuracyof 91.3% for OTA non-incumbent detection.

The energy saving results via ML-enabled rApp control in the the simulated NS-3 environment are encouraging and provide a basis for further enhancement in the ML model as well as the decision-making entity to incorporate other decision variables as the future scope of the work. Also, other prediction models can be implemented to analyze different model performances in the end-to-end experimental deployment. Furthermore, the enhanced rApp version provides an overall energy-saving solution to be used for efficient RAN control/management, not only in experimental simulations but also in any real-world environment.

