\section{Conclusions}
This work aimed to develop and evaluate an Energy Saving Solution as a rApp for the O-RAN architecture. 
The approach consisted of using using variosu componenst to implement our solution, from LSTM fopr trafffic prediction and various logical algorithms for other compoennts. 
The novel part of our proposed solution involved using a Digital Twin to validate the proposed solution in a simulated environment, to prevent causing iussues which cannot be reveresed.
A LSTM neural network model was trained on various time-series datasets. 
\textcolor{red}{[YOGESH] Will complete after results are ready. \\}

%Also, 5G NR data was capturedusing the ESC sensor node with the SDR based CBSD asthe signal source. 
%The collected 5G NR data was mixed withthe NIST generated datasets to train the ML model for 5GNR non-incumbent detection. 
%The model training results show better performance at higher SNR values as expected.
%The highest prediction accuracy of 95.83\% was achieved for \textcolor{blue}{[PULAK - Edit in the end] dataset with signals in the SNR range 40-50 dB}. OTA incum-bent detection is achieved by transmitting the signals in thedataset OTA and results are presented. In the OTA resultsfor incumbent detection, the highest accuracy achieved is 85.35\%. Additionally, the results for the OTA non-incumbent(5G-NR) signal is presented. The model achieves an accuracyof 91.3% for OTA non-incumbent detection.

The energy saving results via ML-enabled rApp control in the the simulated NS-3 environment are encouraging and provide a basis for further enhancement in the ML model as well as the decision-making entity to incorporate other decision variables as the future scope of the work. 
The results indicate that the proposed solution can be a viable option for operators to reduce their OPEX while maintaining the QoS for their subscribers.
Further work can include impelmenting other prediction models to analyze different model performances in the end-to-end experimental deployment. 
Furthermore, the enhanced rApp version provides an overall energy-saving solution to be used for efficient RAN control/management, not only in experimental simulations but also in any real-world environment.
