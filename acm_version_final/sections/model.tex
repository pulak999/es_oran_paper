\section{Design Rationale}
\label{sec:model}

The standalone application for an ESC node described in 2.2 is connected to the OpenSAS. This application indepen-dently senses the CBRS spectrum for any activity. If activityis detected, it sends IQ data to the model running insidethe OpenSAS for incumbent detection. The current imple-mentation is to detect incumbent (radar) in a 5G New Radio(NR) based CBRS network deployment. Additionally, the re-searchers could use this platform to experiment with theirown models for detecting signals of their interest throughthe ESC node in testbed environments.

Network traffic prediction has always been a largely explored subject in networking, with a flurry of recent proposals ushered in by the recent development of machine and deep learning tools. Such deep learning-based algorithms have recently been explored to find potential representations of network traffic flows for all types of networks, including Internet, cellular, etc. We first categorize cellular traffic problems into two main types – temporal prediction problems and spatiotemporal prediction problems. Modelling the traffic flow through a node exclusively as a time series is an example of the temporal approach towards network traffic prediction [11]. High traffic on a given node in a cellular network often implies a high load on the other nearby nodes. Taking the traffic flow of nearby nodes and other external factors into consideration when modelling is known as the spatiotemporal approach to network traffic prediction. Spatiotemporal approaches are found to give slightly more accurate forecasts [12].

Both types of problems can be formulated as supervised learning problems with a difference being in the form of feature representation. In the temporal approach, the collected traffic data can be represented as a univariate time series and the prediction for the values in the future time steps is based on the historical data of the past time steps. In [13], Clemente et Al used Naive Bayes classification and the Holt-Winters method to perform the temporal network forecasting in real time Clemenete et Al first performed systematic preprocessing to reduce bias by selecting the cells with less missing data occurrences, which was then selected to train the classifies to allocate the cells between predictable and non- predictable, taking into account previous traffic forecast error. 

Building upon the temporal approach, Zhang et al. [14] presented a new technique for traffic forecasting that takes advantage of the tremendous capabilities of a deep convolutional neural network by treating traffic data as images. The spatial and temporal variability of cell traffic is well captured within the dimensions of the images. The experiments show that our proposed model is applicable and effective. Even with the ease of machine learning implementations, regression based models have been found to be fairly accurate, as proven by Yu et Al in [15]. In [15], Yu et Al applied a switching ARIMA model to learn the patterns present in traffic flow series, where the variability of duration is introduced and the sigmoid function describes the relation between the duration of the time series and the transition probability of the patterns. The MGCN-LSTM model, presented in [16] by Len et Al, was a spatial-temporal traffic prediction model which implemented a multi-graph convolutional network (MGCN) to capture spatial features, and a multi-channel long short-term memory (LSTM) to recognise the temporal patterns among short-term, daily, and weekly periodic data. The proposed model was found to greatly outperform commonly implemented algorithms such as ARIMA, LSTM and ConvLSTM.

Hybrid models can handle a variety of data types and structures, making them ideal for diverse applications along with combining the best features of different methodologies. This very principle is proven by Kuber et Al in [17] which proposes a linear ensemble model composed of three separate sub-models. Each sub-model is used to predict the traffic load in terms of time, space and historical pattern respectively, handling one dimension particularly. Different methodologies such as time series analysis, linear regression and regression tree are applied to the sub-models, which is aggregated and found to perform comparable to a ResNet-based CNN model. Another approach for the same is highlighted in [18] Tian et Al. The approach involves analysing the chaotic property of network traffic by analyzing the chaos characteristics of the network data. [18] proposes a neural network optimization method based on efficient global search capability of quantum genetic algorithm and based on the study of artificial neural networks, wavelet transform theory and quantum genetic algorithm. The proposed quantum genetic artificial neural network model can predict the network traffic more accurately compared to a similarly implemented ARMA model.\\

\subsection{Model Selection}
Model Selection. \textcolor{blue}{Why did we select the model we did?}

Model will handle missing data points during training and inferencing and raise appropriate flags

\subsection{Data Selection}
Data Selection. \textcolor{blue}{How did we select data to train our model on? How did we know the model will work with less data?}

\subsection{Performance Metrics}
Performance Metrics. \textcolor{blue}{Might not be needed if explained well in previous section.}
