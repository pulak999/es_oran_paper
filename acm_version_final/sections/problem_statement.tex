\section{Problem Statement}
\label{sec:ps}

In case of dense urban deployments, where the greatest number of sites are commissioned, each coverage sector has multiple carriers deployed. 
This is done for several reasons, mainly to improve the capacity at the site or even to take advantage of different characteristics of different carriers (low-band for coverage, mid-band or high-band for capacity improvements). 
Additionally, coverage overlap is provided across the sites to improve call drop and handover failures. 
While complex techniques are used to balance the users and traffic across these carriers locally at the site, the aggregate traffic carried by a regional network is lesser than the total capacity of a properly planned network even during high load situations. 
During off-traffic hours, these networks tend to be further underutilized. The energy $E$ consumed by each node is modeled \textcolor{blue}{[CITE]} as a sum of two parts, the baseline energy consumption and the energy consumed due to traffic:

\begin{equation}
E_{\text{total}} = E_{\text{quiescent}} + \gamma N_{\text{traffic}} \tag{1}
\end{equation}

Here $E_{\text{quiescent}}$ represents the energy consumed by the system when it's in a quiescent or idle state. 
This is the baseline energy consumption that occurs regardless of the level of traffic and it depends on the operating point of the radio and server.
$\gamma$ represents the additional energy consumed per unit of traffic, represented as $N_{\text{traffic}}$.

In order to minimize the effects of distortion due to high Peak to Average Power Ratio (PAPR) of orthogonal frequency division access (OFDMA) waveforms, the operating point is so maintained that we have considerable amount of current dissipated even when there are no users latched on the network. 
Hence, turning off the radios is an effective way of addressing this loss. 
However, this seemingly simple decision has multiple aspects to be looked at, the most important ones being:

\begin{itemize}
\item Which carriers should be switched off to achieve optimum power saving without impacting the QoS?
\item What is the right time to switch off or on these carriers?
\end{itemize}

Although these are well studied problems, most of the implementations and studies are based on heuristics. \textcolor{blue}{CITE}
While heuristics can speed up the problem-solving process, they may not always provide the most accurate or optimal solution. 
%/begin{comment}[\textcolor{blue}{CITE}]/end{comment}
This paper formulates the problem in a way that it can be solved with closed loop methods with finite time algorithms. 
We also define a metric based on statistical techniques to evaluate the “goodness” of any algorithm which attempts to implement energy saving through this or other techniques. 

\begin{comment}
O-RAN defines rApps [O-RAN Alliance, "O-RAN Architecture Description," O-RAN.WG1.O-RAN-Architecture-Description-v03.00, July 2020.] as modular applications designed to consume and/or produces non real time management and automation services for the orchestration and optimization of network resources and operations. 
The non-RT RIC, in particular, is designed to handle tasks that do not require immediate response. This makes it ideal for applications focused on long-term optimization and strategic planning, such as energy control. 
% what is SMO
% little intro to the rApp
The rApp receives input data from the Radio Database, Traffic Predictor, and Coverage Predictor, and sends a policy, sent as a declarative statement and across the A1 interface, to the Near-RT RIC over what to do. 
The decision is made periodically, with a 1-hour prediction window and 15-minute slots, i.e., four predictions are made every window. 
The rApp is designed to be shared across multiple rApps and can import data from RF link simulators and drive tests through an external interface. 
A Dashboard for visualization of the Radio Mapping Database is also used as shown in the given figure /begin{comment}[\textcolor{blue}{CITE}]/end{comment}. \\

The rApp is data driven in the sense that it does not incorporate a rules-based logic but determines the rules which meet the target objective based on the input data and network configuration. 

How does this link to the TS xApp? (receive a policy from the rApp)  \\

Space-time partitioning: This is a technique used to divide data based on both spatial (location) and temporal (time) dimensions. In the context of the rApp, this could involve organizing Key Performance Indicators (KPIs) by specific geographical areas (cells, sectors, etc.) and time periods to better manage and analyze the data.

Continuous time-based aggregation: This refers to the process of continuously collecting and summarizing data over time. Instead of analyzing data at discrete intervals, it is aggregated in a continuous manner, which allows for more fluid and accurate monitoring of KPIs.

Group KPIs by time: This involves organizing the Key Performance Indicators (KPIs) into groups based on the time they were recorded. This helps in analyzing trends and patterns over specific time periods.
\end{comment}