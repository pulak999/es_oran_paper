\section{Conclusions}
This work aimed to develop and evaluate an Energy Saving Solution as a rApp for the O-RAN architecture. 
The approach consisted of using using various components to implement our solution.
The novel aspect of our approach lies in the use of a Digital Twin to validate the proposed solution using KL Divergence as a statistical measure to accurately quantify the changes in our system.
We leveraged machine learning to identify patterns in network traffic, rather than attempting to construct a fully accurate model of the network traffic.
The decision-making entity and coverage predictor operated based on cellular-informed logical rules.

The energy saving results via ML-enabled rApp control in the the simulated NS-3 environment are encouraging and provide a basis for further enhancement in the ML model as well as the decision-making entity to incorporate other decision variables as the future scope of the work. 
The results indicate that the proposed solution can be a viable option for operators to reduce their OPEX while maintaining the QoS for their subscribers.
Further work can include impelmenting other prediction models to analyze different model performances in the end-to-end experimental deployment. 
Furthermore, the enhanced rApp version provides an overall energy-saving solution to be used for efficient RAN control/management, not only in experimental simulations but also in any real-world environment.
