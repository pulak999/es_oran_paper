\section{Energy-Saving Solution}
\subsection{Overview}
\label{sec:overview}

The solution to address the three aspects of the problem discussed in the previous section comprises of several components. 
\hyperref[fig:sol]{Figure 2} outlines a simplified version of how the components interact with each other to implement our ES solution. 
The data at each stage of the operational flow is indicated by numbers (1..6). 

The system is fed with the key performance indicators (KPI) data captured from the network (1) along with topology and configuration information (2). 
The former is consumed by the traffic predictor to determine if in the upcoming hours, the volume of the traffic changes by an amount necessary to relook at the current state of all the nodes. 
The topology and configuration data (2) is static information usually gathered from the Element Management System (EMS). 
The decision algorithm relies on the predicted traffic estimates (3) and the overlap predictions (5) to determine the state of the network, i.e., which nodes should be turned on and which should be off. 

However, at this stage, we need to make sure that such a decision is not detrimental to the performance of the network. 
Hence, it is run by the Digital Twin of the network to evaluate the “goodness” and if it passes a threshold, it is presented to the actual control unit, the EMS, for execution. 
Otherwise, we reevaluate the decision. 

Each of the components are complex systems and described in the following sections.
The rApp is data driven in the sense that it does not incorporate a rigid logic for cell shutdown, but instead determines the rules which meet the target objective based on the input data and network configuration. 
The non-RT RIC, in particular, is designed to handle tasks that do not require immediate response. 
This makes it ideal for applications focused on long-term optimization and strategic planning, such as energy control. 

%\definecolor{mygrey}{RGB}{145,146,156}
\definecolor{mypurple}{RGB}{153,102,204}
\definecolor{myotherblue}{RGB}{0,191,255}
\definecolor{mygreen}{RGB}{144,238,144}
\definecolor{myyellow}{RGB}{255,215,0}
\definecolor{myorange}{RGB}{255,165,0}
\definecolor{myothergreen}{RGB}{35,79,30}
\definecolor{myred}{RGB}{255,127,80}
\definecolor{myindigo}{RGB}{100,110,230}

\begin{figure}
    \centering
    \begin{tikzpicture}[
        subox1/.style={draw, fill=myorange, minimum width=1.1cm, minimum height=0.7cm, font=\footnotesize},
        subox2/.style={draw, fill=myindigo, minimum width=0.55cm, minimum height=0.7cm, font=\footnotesize},
        subox3/.style={draw, fill=myred, minimum width=0.55cm, minimum height=0.7cm, font=\footnotesize},
        subox4/.style={draw, fill=myothergreen, minimum width=2.4cm, minimum height=0.4cm, font=\footnotesize},
        box1/.style={draw, fill=mygreen, minimum width=2.6cm, minimum height=1.5cm, font=\footnotesize},
        box2/.style={draw, fill=myyellow, minimum width=1.7cm, minimum height=1.5cm, font=\footnotesize},
        smallbox1/.style={draw, fill=mypurple, minimum width=0.85cm, minimum height=0.5cm, font=\footnotesize, rounded corners},
        smallbox2/.style={draw, fill=myotherblue, minimum width=0.85cm, minimum height=0.5cm, font=\footnotesize, rounded corners},
        smallbox3/.style={draw, fill=mygrey, minimum width=0.85cm, minimum height=0.5cm, font=\footnotesize, rounded corners},
        flatbox1/.style={draw, fill=myotherblue, minimum width=2cm, minimum height=0.25cm, font=\footnotesize},
        flatbox2/.style={draw, fill=myotherblue, minimum width=3.2cm, minimum height=0.25cm, font=\footnotesize},
        flatbox3/.style={draw, fill=mygrey, minimum width=3.2cm, minimum height=0.35cm, font=\footnotesize},
        bigbox1/.style={draw, fill=myotherblue, minimum width=0.8cm, minimum height=0.7cm, font=\footnotesize},
        bigbox2/.style={draw, fill=mygrey, minimum width=0.75cm, minimum height=0.7cm, font=\footnotesize},
        bigbox3/.style={draw, fill=mygrey, minimum width=4.45cm, minimum height=0.4cm, font=\footnotesize},
        hugebox/.style={draw, , fill=mygrey, minimum width=9cm, minimum height=0.75cm, font=\footnotesize},
        scale=0.4 % Adjust this value to fit in one or two columns
    ]
    
    % Base layer
    \node[hugebox, text=white] (ns3) {\textbf{\texttt{ns-3 (network simulator)}}};

    % Second layer
    \node[smallbox1, text=white, above left=0.2 and -1.5 of ns3] (sdnr) {\texttt{SDNR}};
    \node[flatbox1, text=white, above left=0.2 and -4.5 of ns3] (o1) {\texttt{O1-adapter}};
    \node[flatbox2, text=white, above right=0.2 and -3.2 of ns3] (e2) {\texttt{E2-adapter, USOI}};
    \node[flatbox3, text=white, above right=0.75 and -3.2 of ns3] (near-ric) {\textbf{\texttt{Near-RT RIC}}};
    
    % Third layer
    \node[bigbox1, text=white, above right=1.25 and -0.95 of ns3, align=center] (other-x) {\texttt{other} \\ \texttt{xApps}};
    \node[bigbox2, text=white, above right=1.25 and -2.27 of ns3, align=center] (kpi-x) {\textbf{\texttt{KPI-MON}} \\ \textbf{\texttt{xApp}}};
    \node[bigbox2, text=white, above right=1.25 and -3.15 of ns3, align=center] (ts-x) {\textbf{\texttt{TS}} \\ \textbf{\texttt{xApp}}};
    
    \node[bigbox3, text=white, above left=0.85 and -4.5 of ns3, align=center] (smo) {\textbf{\texttt{SMO}}};
    \node[bigbox3, text=white, above left=1.4 and -4.5 of ns3, align=center] (non-ric) {\textbf{\texttt{Non-RT RIC}}};

    % Fourth layer
    \node[box1, above left=0.15 and -2.6 of non-ric] (es) {};
    \node[box2, text=white, above left=0.15 and -4.45 of non-ric, align=center] (rapps) {\texttt{other} \\ \texttt{rApps}};
    
    % Super-Impose layer
    \node[subox1, text=white, above left=0.3 and -1.2 of non-ric, align=center] (dt) {\textbf{\texttt{DT}}};
    \node[subox2, text=white, above left=0.3 and -1.85 of non-ric, align=center] (cp) {\textbf{\texttt{CP}}};
    \node[subox3, text=white, above left=0.3 and -2.5 of non-ric, align=center] (tp) {\textbf{\texttt{TP}}};
    \node[subox4, text=white, above left=1.1 and -2.5 of non-ric, align=center] (dme) {\textbf{\texttt{DME}}};
    \node[smallbox1, text=white, above left=0.55 and -3.9 of ns3, align=center] (ves) {\texttt{VES} \\ \texttt{clctr}};
    
    % Arrows

    %\draw[-, line width=0.5mm] (sdnr.south) -- (ns3.north -| sdnr.south);
    %\draw[-, line width=0.5mm] (o1.south) -- (ns3.north -| o1.south);
    %\draw[-, line width=0.5mm] (e2.south) -- (ns3.north -| e2.south);
    %\draw[-, line width=0.5mm] (near-ric.south) -- (e2.north -| near-ric.south);
    %\draw[->, line width=0.5mm] (non-ric.south) -- (e2.north -| non-ric.south);
    
    % Text
    %\node[rotate=90, red, anchor=south, font=\tiny] at ($(es.west)+(-0.4,0)$) {Cell ON/OFF};
    \node[above=0.01 of es] {\small \textbf{\texttt{ES rApp}}};

    \end{tikzpicture}
    \caption{System Architecture Diagram}
    \label{fig:system-architecture}
\end{figure}
    

\subsection{Base Station Control Algorithm}

\begin{algorithm} [t!]
    \caption{
        \texttt{\sysname pipeline flow}, 
        % \textit{Input:} 
        %     packet \textsf{pkt},
        %     current state \textsf{curr\_state},
        %     current round number \textsf{curr\_rd},
        %     operation mode \textsf{op\_mode},
        % \textit{Output:} 
        %     enncrypted/decrypted pkt \textsf{pkt'} 
    }
        \begin{algorithmic} [1]
            \Procedure{p4ead\_pipeline}{\textsf{pkt}, \textsf{curr\_state}, \textsf{curr\_rd}, \textsf{op\_mode}}
                \State prnd\_count $\gets$ 0
                \While{\textsf{prnd\_count} $<$ \textsf{RPP}} \label{algline:rpp}
                    \If{curr\_state == START}
                        \State curr\_state $\gets$ INIT
                        \State \textsf{pkt} $\gets$ \Call{INIT}{\textsf{pkt}}
                    \ElsIf{curr\_state == INIT}
                        \If{curr\_rd == 12}
                            \State curr\_state $\gets$ ABS\_AD
                            \State \textsf{pkt} $\gets$ \Call{AD\_ABS}{\textsf{pkt}}
                            \State curr\_rd $\gets$ 0
                        \EndIf
                    \ElsIf{curr\_state == ABS\_AD}
                        \If{curr\_rd == 6}
                            \State curr\_state $\gets$ ABS\_IP
                            \State \textsf{pkt} $\gets$ \Call{IP\_ABS}{\textsf{pkt}}
                            \State curr\_rd $\gets$ 0
                        \EndIf
                    \ElsIf{curr\_state == ABS\_IP}
                        \If{curr\_rd == 6}
                            \State curr\_state $\gets$ FINAL
                            \State curr\_rd $\gets$ 0
                        \EndIf
                    \ElsIf{curr\_state == FINAL}
                        \If{curr\_rd == 12}
                            \State curr\_state $\gets$ END
                            \State \textsf{pkt} $\gets$ \Call{TAG}{\textsf{pkt}}
                            \State \Call{break}{}   
                        \EndIf
                    \EndIf
                    \State \textsf{pkt} $\gets$ \Call{p\_rnd}{\textsf{pkt}}  \Comment{do one P-RND}
                    \State prnd\_count $\gets$ prnd\_count $+ 1$
                    \State curr\_rd $\gets$ curr\_rd $+ 1$
                \EndWhile
                \If{curr\_state == END}
                    \If{op\_mode == DECRYPT}
                        \State valid\_tag $\gets$ \Call{verify\_tag}{\textsf{pkt}}
                        \If{$\neg$valid\_tag}
                            \State \Call{drop}{\textsf{pkt}}
                        \EndIf
                    \EndIf
                    \State \textsf{pkt'} $\gets$ \textsf{pkt}
                    \State \Call{forward}{\textsf{pkt'}}
                \Else
                    \State \Call{recirculate}{\textsf{pkt}} 
                \EndIf
            \EndProcedure
        \end{algorithmic}
        \label{alg:pipeline}
    \end{algorithm}
    
     % \If {(cCount - \textsf{cms}.get\_count(flowId)) > $\sigma$} \label{algorithm_access_check_count_start}
     %                \State \textsf{pkt}.ctrl $\gets 0$ \COMMENT{prepare alert}
     %                \State \textsf{pkt}.dst\_ip $\gets$ \Call {Get\_Border\_IP}{\textsf{pkt}.ctrl.borderId}\label{algorithm_access_check_count_end}

     \begin{comment}
        \begin{algorithm} [H]
        \caption{Coverage Predictor}
        \label{alg:control_application}
        \SetKwInOut{Input}{Input}
        \SetKwInOut{Output}{Output}
        
        \Input{DecisionVariables}
        
        \SetKwProg{Fn}{Function \emph{applyControl}}{}{end}
        \Fn{}{
            delay $\gets$ EWMA(delay, delay_in)\;
            
            \While{True}
            {
                \If{$condition1$}
                {
                    \Return{$CellBringup$}\;
                }
                \ElseIf{$condition2$}
                {
                    \Return{$CellShutdown$}\;
                }
                \Else
                {
                    $NoChange$\;
                }
            }
        }
        
        \Output{$Control\_Action$} % Adjust the output based on your actual output
        
        \end{algorithm}
        \end{comment}


        \begin{comment}
            \begin{algorithm} [H]
            \caption{Control Application to the RAN Stack}
            \label{alg:control_application}
            \SetKwInOut{Input}{Input}
            \SetKwInOut{Output}{Output}
            
            \Input{DecisionVariables}
            
            \SetKwProg{Fn}{Function \emph{applyControl}}{}{end}
            \Fn{}{
                delay $\gets$ EWMA(delay, delay_in)\;
                
                \While{True}
                {
                    \If{$condition1$}
                    {
                        \Return{$CellBringup$}\;
                    }
                    \ElseIf{$condition2$}
                    {
                        \Return{$CellShutdown$}\;
                    }
                    \Else
                    {
                        $NoChange$\;
                    }
                }
            }
            
            \Output{$Control\_Action$} % Adjust the output based on your actual output
            
            \end{algorithm}
            
            \begin{algorithm}[H]
            \caption{Cell Shutdown Procedure}
            \label{alg:cellshut}
            \SetKwProg{generate}{Function \emph{CellShutdown}}{}{end}
            
            Map store=new Map(obj, queue)\;
            \generate{Object pivot}{
                 \ForAll{child $c$ in pivot}{
                 \If{ $c$'s FieldContext is not set and $c$ is fusible}{
                      generate($c$)\;
                  }
                 }
                 build pivot's fieldContext $fc$\;
                 EmitClassName\;
                 EmitFields($fc$)\;
                 EmitMethods($fc$)\;
            }
            \end{algorithm}
            
            \begin{algorithm}
            \caption{Cell Shutdown Procedure}
            \SetAlgoLined
            \KwData{Lat-long of all the transmitters (antenna characteristics, antenna height, etc.)}
            \KwData{Field strength at any point from all the transmitters (using CloudRF)}
            \KwData{Entire geography subdivided into smaller areas (30m x 30m) each represented by a pixel obtained from CloudRF map}
            \For{each pixel}{
                Find the RX power from each BS (more than a certain threshold)\;
                Find the strongest BS\;
                Calculate SINR = $\frac{\text{power of strongest cell}}{\text{sum of all remaining cells' power + noise power}}$\;
            }
            \For{each BS}{
                Find the list of neighbors\;
                Find the count of pixels where:\;
                \quad I. Serving BS is dominant\;
                \quad II. Neighbor is within a threshold of the serving cell\;
                This count is degree of overlap $C_{ij}$\;
                Total overlap is $C_{ij} + C_{ji}$\;
                Weight of each edge of the undirected graph is the degree of overlap\;
                Calculate the rank $M_i = \frac{K \cdot \sum \text{degree of the node}}{\text{Traffic volume at node (i)}}$ (let $K=1$ for now)\;
            }
            Sort the list of BS in order of descending rank; the top one is the candidate for shutdown\;
            Select the top candidate in rank list above for shutdown\;
            With this candidate being shutdown (power = 0), find the SINR distribution (in digital twin)\;
            \If{the SINR distribution is within bounds}{
                Break the loop\;
            }
            \Else{
                Continue to step 1\;
            }
            \end{algorithm}
            
            \begin{algorithm}
            \caption{Cell Bringup Procedure}
            \SetAlgoLined
            \KwData{For each cell we have desired pattern for CQI distribution for each traffic pattern and time of day}
            \For{each cell}{
                Measure the distance between CQI distributions observed and desired\;
                \If{the observed is off by a certain threshold}{
                    Turn the cell ON\;
                    Recalculate the predicted CQI distribution\;
                    \If{the distribution improves}{
                        Keep the cell ON\;
                    }
                    \Else{
                        Leave the cell OFF\;
                    }
                }
            }
            \end{algorithm}
            
            \begin{algorithm}
                \SetKwInOut{Input}{Input}
                \SetKwInOut{Output}{Output}
            
                \underline{function CellBringup} $()$\;
                \Input{Two nonnegative integers $a$ and $b$}
                \Output{$\gcd(a,b)$}
                \eIf{$b=0$}
                  {
                    return $a$\;
                  }
                  {
                    return Euclid$(b,a\mod b)$\;
                  }
                \caption{Algorithm}
            \end{algorithm}
            \end{comment}
            
        
