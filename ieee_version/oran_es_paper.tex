\documentclass[conference]{IEEEtran}
\IEEEoverridecommandlockouts
% The preceding line is only needed to identify funding in the first footnote. If that is unneeded, please comment it out.
\usepackage{cite}
\usepackage{comment}
\usepackage{hyperref}
\usepackage{graphicx}
\usepackage{textcomp}
\usepackage{xcolor}
\usepackage{subcaption}
\usepackage{multirow}
\usepackage{booktabs}
\usepackage{caption}
\usepackage{tabularx}
\usepackage{ragged2e}
\usepackage{makecell}
\usepackage{array}
\usepackage{xcolor}
\usepackage{adjustbox}
\usepackage[linesnumbered,ruled,vlined]{algorithm2e}
\usepackage{amsmath}

\def\BibTeX{{\rm B\kern-.05em{\sc i\kern-.025em b}\kern-.08em
    T\kern-.1667em\lower.7ex\hbox{E}\kern-.125emX}}
\begin{document}

\title{\textcolor{blue}{[WORKSHOP]}Towards Greener Networks: RApp-Based Cell Control over O-RAN Deployments}

\author{\IEEEauthorblockN{1\textsuperscript{st} Given Name Surname}
\IEEEauthorblockA{\textit{dept. name of organization (of Aff.)} \\
\textit{name of organization (of Aff.)}\\
City, Country \\
email address or ORCID}
}

\maketitle

\begin{abstract}
Abstract.
\end{abstract}

\begin{IEEEkeywords}
Keywords.
\end{IEEEkeywords}

\section{Extra}

- Blue : Things to Highlight/Expand \\
- Red : Questions to ask \\

\section{Introduction}
Introduction.
\\
- The objective of this paper is to describe an rApp hosted by a Non-Real-Time RIC which controls a set of gNBs in a 5G network, which has the capability of reducing the network energy consumption during low traffic period by selectively shutting down specific sectors. The document describes the design of rApp and the requirements in terms of input data and the controls required over the network element for proper functoning of the rApp. It does not dwell into the details of the plaTorms hos+ng the rApp or the network func+ons or simula+ons thereof which provide data to the rApp. \\
- We use simple time-based thresholds to determine WHEN to turn the cell(s) OFF/ON i.e, on a day if at XX:XX hrs, if traffic volume is less than XXX Mbps, turn the cell off or otherwise \\
- Proposes turning ON/OFF a cell based on “Degree of overlap” of coverage. Cells with higher degree of overlap can be served with overlapping cells hence qualify earlier. Cross-validation is done by building a digital twin of the underlying network and implement the decision (ON/OFF) on it to validate the metrics of significance for network operator: 1) Spectral efficiency, 2) CQI distributions, 3) Active users, 4) NW Throughput \\
- While a mobile network consists of multiple parts, the radio access network (RAN) is responsible for most of the energy consumption in a mobile network. ==> 
Green Future Networks – Sustainability Challenges and Initiatives in Mobile Networks by NGMN Alliance,
December 2021
\href{https://www.ngmn.org/wp-content/uploads/210719_NGMN_GFN_Sustainability-Challenges-andInitiatives_v1.0.pdf} \\
- A radio access network comprises of several cell sites with site infrastructure equipment and base station equipment. \\
- O-RAN can contribute to this effort immensely. Its disaggregated and virtualized architecture adds complexity; however, energy is the next major challenge O-RAN must overcome. \\
- Rimedo Labs recently released Energy Saving rApp (ES-rApp) for Non-RT RIC focusing on Massive MIMO use cases. \\ 
\\
- The RF carrier shutdown feature (typically hosted by the SMO and non-RT RIC in O-RAN architecture)
periodically checks the service load of multiple carriers and if the service load is below a specified threshold, the capacity-layers are shut off (see Figure 12). The UEs served by those carriers can camp on or access the services from the carrier providing basic coverage. When the load of the carrier providing basic coverage is higher than a specified threshold, the base station turns on the carriers that have been shut down for service provisioning. \\
- RF carriers can be shut down by non-RT RIC rAPPs more intelligently using information from telemetry data from E2 interface with O-RAN.\\
- Switching off under-loaded cells during network operation without affecting the user experience (call drops, QoS degradation, etc.) is one way to achieve RAN energy efficiency. \\
- A typical energy savings scenario is realized when capacity booster cells are deployed under the umbrella of cells providing basic coverage and the capacity booster cells are switched off to enter dormant mode when its capacity is no longer needed and reactivated on a need basis. \\
- When the booster gNB with CU/DU split decides to switch off cell(s) to the dormant state, the decision is typically made by the gNB-DU based on cell load information or by the OAM entity (non-RT RIC in O-RAN architecture). Before the cell in the gNB-DU enters into the dormant mode, the gNB-DU will send the gNB-DU configuration update message to the gNB-CU to indicate that the gNB-DU will switch off the cell subsequently sometime later. \\ 
- \textcolor{blue}{During the switch-off period, the gNB-CU offloads the UE to a neighboring cell and simultaneously will not admit any incoming UE to this cell being switched-off. Is this load balancing performed?}\\
- https://networkbuilders.intel.com/solutionslibrary/a-holistic-study-of-power-consumption-and-energy-savings-strategies-for-open-vran-systems \\ 
\\
- RF channel switch off/on \\
- \textcolor{blue}{However, the switch off/on decisions are need a lot of KPIs reporting and efficient actions so that guarantee the overall user experience. Also, there are conflicting targets between system performance and energy savings.} \\
\\
- \textcolor{blue}{Offline learning is normally preferred (including reinforcement learning which is usually performed online) due to the nature of the network environment, which is prone to misconfiguration and errors leading to outages. The proposed approach is for the model to first be trained with offline data and that the trained model then be deployed in the network for inference.} 
- \textcolor{blue}{ES-rApp aims at minimizing the overall energy consumption by switching off cells that are not loaded too much, and turning on the cells if the traffic load goes high.}\\
- \textcolor{blue}{If the shift of all the users from a given cell is possible (i.e., if the performance requirements of already served users can be guaranteed), it generates the appropriate policy and sends it via the A1 interface using A1-P to the Near-RT-RIC.} \\
- First, the E2 Nodes are configured by the Service Management and Orchestration (SMO) to report the data necessary for energy-saving algorithms via the O1 Interface to the Collection and Control unit. Assuming that the Non-RT RIC and SMO are tightly coupled the NonRT RIC retrieves the collected data through internal SMO communication \textcolor{green}{how???}. The O-RUs are involved in this use case. The E2 Nodes need to configure them to report data through the Open RAN Fronthaul Management Plane (Open FH M-Plane) interface.\\
- Before switching off/on carrier(s) and/or cell(s), the E2 Node may need to perform some preparation actions for off switching (e.g. check ongoing emergency calls and warning messages, to enable, disable, modify Carrier Aggregation and/or Dual Connectivity, to trigger HO traffic and UEs from cells/carriers to other cells or carriers, informing neighbour nodes via X2/Xn interface etc.) as well as for on switching (e.g., cell probing, informing neighbour nodes via X2/Xn interface etc.). \\

Network traffic prediction has always been a largely explored subject in networking, with a flurry of recent proposals ushered in by the recent development of machine and deep learning tools. Such deep learning-based algorithms have recently been explored to find potential representations of network traffic flows for all types of networks, including Internet, cellular, etc. We first categorize cellular traffic problems into two main types – temporal prediction problems and spatiotemporal prediction problems. Modelling the traffic flow through a node exclusively as a time series is an example of the temporal approach towards network traffic prediction [11]. High traffic on a given node in a cellular network often implies a high load on the other nearby nodes. Taking the traffic flow of nearby nodes and other external factors into consideration when modelling is known as the spatiotemporal approach to network traffic prediction. Spatiotemporal approaches are found to give slightly more accurate forecasts [12].

Both types of problems can be formulated as supervised learning problems with a difference being in the form of feature representation. In the temporal approach, the collected traffic data can be represented as a univariate time series and the prediction for the values in the future time steps is based on the historical data of the past time steps. In [13], Clemente et Al used Naive Bayes classification and the Holt-Winters method to perform the temporal network forecasting in real time Clemenete et Al first performed systematic preprocessing to reduce bias by selecting the cells with less missing data occurrences, which was then selected to train the classifies to allocate the cells between predictable and non- predictable, taking into account previous traffic forecast error. 

Building upon the temporal approach, Zhang et al. [14] presented a new technique for traffic forecasting that takes advantage of the tremendous capabilities of a deep convolutional neural network by treating traffic data as images. The spatial and temporal variability of cell traffic is well captured within the dimensions of the images. The experiments show that our proposed model is applicable and effective. Even with the ease of machine learning implementations, regression based models have been found to be fairly accurate, as proven by Yu et Al in [15]. In [15], Yu et Al applied a switching ARIMA model to learn the patterns present in traffic flow series, where the variability of duration is introduced and the sigmoid function describes the relation between the duration of the time series and the transition probability of the patterns. The MGCN-LSTM model, presented in [16] by Len et Al, was a spatial-temporal traffic prediction model which implemented a multi-graph convolutional network (MGCN) to capture spatial features, and a multi-channel long short-term memory (LSTM) to recognise the temporal patterns among short-term, daily, and weekly periodic data. The proposed model was found to greatly outperform commonly implemented algorithms such as ARIMA, LSTM and ConvLSTM.

Hybrid models can handle a variety of data types and structures, making them ideal for diverse applications along with combining the best features of different methodologies. This very principle is proven by Kuber et Al in [17] which proposes a linear ensemble model composed of three separate sub-models. Each sub-model is used to predict the traffic load in terms of time, space and historical pattern respectively, handling one dimension particularly. Different methodologies such as time series analysis, linear regression and regression tree are applied to the sub-models, which is aggregated and found to perform comparable to a ResNet-based CNN model. Another approach for the same is highlighted in [18] Tian et Al. The approach involves analysing the chaotic property of network traffic by analyzing the chaos characteristics of the network data. [18] proposes a neural network optimization method based on efficient global search capability of quantum genetic algorithm and based on the study of artificial neural networks, wavelet transform theory and quantum genetic algorithm. The proposed quantum genetic artificial neural network model can predict the network traffic more accurately compared to a similarly implemented ARMA model.\\

- This implementation is akin to a traffic steering xApp, as it involves offloading traffic from low-load cells to other cells, ensuring that as many RUs as possible can enter a low-power sleep mode. The handovers and traffic redistribution are predicted and managed in real-time by the xApp, enhancing the overall energy efficiency of the network. ==> https://www.diva-portal.org/smash/get/diva2:1765998/FULLTEXT01.pdf \\

- Ericcson, Rimodo Labs, Juniper Networks, VMWare have preexisiting rApps on the market, but have not described them very well. \\

- During the ES-rApp operation, it gets information from the O1 interface about cells available in the network, their type (macro cell or small cell), and the PRB usage of each cell. ES-rApp collects such data during predefined times and calculates average O-DU PRB usage in the time domain. Periodically, the rApp can make decisions about enabling or disabling one of the cells. ES-rApp will enable a cell in case of congestion (high PRB usage) observed in at least one cell that is currently enabled. ES-rApp will disable a cell in case of average PRB usage below some threshold. If none of the situations occurs, the ES-rApp continues observation. ==> https://rimedolabs.com/blog/es-rapp-control-over-ts-xapp-in-oran/ \\

- VMWare and VIAVI ==> https://www.virtualexhibition.o-ran.org/classic/generation/2024/category/intelligent-ran-control-demonstrations/sub/intelligent-control/394 \\

- Juniper's implementation ==> https://blogs.juniper.net/en-us/service-provider-transformation/delivering-on-the-o-ran-promise-with-juniper-networks-ran-intelligent-controller-ric \\

*last 1 para for the novelty of our paper compared to existing work. should i write anything here?* \textcolor{red}{\textit{something like: } To the best of our knowledge, comparisons between ARIMA, Prophet, LSTM trained on a single dataset but applied in various conditions have not been well studied in the literature}

\subsection{Key Questions}
\textcolor{blue}{
Key questions to answer to implement above: \\
- Key question 1 - When to turn off and on the cells? \\
- Key question 2 - Which cells to turn on and off? \\
- Key question 3 - what is the goodness measure of particular energy saving decision \\
\\
\textit{How does this link to the TS xApp?}
}

\section{Energy Saving rApp}

\textcolor{blue}{Good intro required here}. The rApp is data driven in the sense that it does not incorporate a rules-based logic but determines the rules which meet the target objective based on the input data and network configuration. A Dashboard for visualization of the Radio Mapping Database is also used. The rApp has the following components: \\

\subsection{Architecture Overview}

\subsubsection{Digital Twin}

Uses CloudRF to determine the coverage. Simulation which takes our parameters if need into context. \textcolor{red}{More information required.} \\

\subsubsection{Radio Database}

The \textbf{Radio Database} is a geospatial database that indexes data using latitude, longitude, and altitude, including clutter information. Currently cloudRF has internal clutter database and we rely on it (it is not exposed outside cloudRF). \textcolor{red}{This database is initialized with network inventory and predicted RF power (downlink) for each pixel from sectors exceeding a predefined threshold (Pth). \textbf{How is threshold decided? - Sensitivity of mobile devices in use. Defined by user.}}. The predictions are generated using a Radio Link budget simulator, using CloudRF. Additionally, the database stores timestamped measurement reports from gNBs in a sliding window with a preconfigured depth (td). Notably, this Radio Database can be external to the rApp, allowing it to be shared across multiple rApps. It also has the capability to import data from RF link simulators and drive tests through an external interface. \\

\subsubsection{Traffic Predictor}

The \textbf{Traffic Predictor} estimates the net traffic volume, percentage PRB utilization, and the number of active UEs for each sector as a function of time. This prediction is based on historical data and previous measurements. The Traffic Predictor employs ARIMA/SERIMA algorithms to forecast these values for the future. Input information for this component is expected in 15-minute intervals. \\

\textcolor{red}{Are LSTMs used here? Refer to Doc2.}

\subsection{Coverage Predictor}

The Coverage Predictor is responsible for predicting the coverage overlap between adjacent sectors. It also updates the link level prediction model based on actual measured values to enhance accuracy. The input to the system is the simulated received power level (obtained from cloudRF) for each pixel from all the sectors with contributions higher than Pth. \textcolor{red}{How is Pth decided?} \\

\textcolor{blue}{Convert to Algo}
Following steps are followed: \\
Step-C1: For each pixel, find the sector which has the highest power to that pixel. This is marked as the serving sector. This step determines the sector boundaries in the network. The other contributors are marked as interfering power. \\
Step-C2: For each sector iden+fied in step C1, find the overlap in coverage with adjacent sectors. This is done using the following steps: \\
	- Step-C2.1: Two sectors Si and Sj are deemed adjacent if there exists a pixel in Si where Sj is the strongest interferer and also if there exists a pixel in Sj where Si is the strongest interferer.
	- Step-C2.2: For all pairs of sectors Si and Sj iden+fied in Step-C2.1, find the count of pixels in Si, where Sj is the strongest interferer and all the pixels in Sj where Si is the strongest interferer. Find the value Eij, which is the sum of all the pixels thus iden+fied.
	- Step-C2.3: Generate the Overlap Graph of the network where each sector Si is the vertex while the Eij computed in Step-C2.2 is the weighted edge

\begin{comment}
\begin{algorithm} [H]
\caption{Coverage Predictor}
\label{alg:control_application}
\SetKwInOut{Input}{Input}
\SetKwInOut{Output}{Output}

\Input{DecisionVariables}

\SetKwProg{Fn}{Function \emph{applyControl}}{}{end}
\Fn{}{
    delay $\gets$ EWMA(delay, delay_in)\;
    
    \While{True}
    {
        \If{$condition1$}
        {
            \Return{$CellBringup$}\;
        }
        \ElseIf{$condition2$}
        {
            \Return{$CellShutdown$}\;
        }
        \Else
        {
            $NoChange$\;
        }
    }
}

\Output{$Control\_Action$} % Adjust the output based on your actual output

\end{algorithm}
\end{comment}


\subsection{Energy Saving Decision Entity}

The \textbf{Energy Saving Decision Entity} utilizes the results from both the Traffic Predictor and the Coverage Predictor to identify sectors in the network that can be shut down with minimal service impact. This entity performs simulations to evaluate the impact of energy-saving decisions before configuring the network through the SMO or EMS. \textcolor{blue}{The functioning of this entity is defined in Algo. 1}

1. For each sector Si, iden+fy sectors Sj, Sk, . . . , Su, which has an edge with Si as found from the Coverage Predictor \\
2. From the historical results of the Traffic Predictor, for each sector Si, find a set of vectors Ri(n) <rj, rk, . . . , ru>, where each component of the vector is the PRB u+liza+on ra+o of the corresponding sector, such that it occurs concurrently and the magnitude of the distance between any two such vectors is greater than defined threshold. \textcolor{red}{What does this mean? What is the point of this being taken?} \\

3. SINR calculation. the strength of the wanted signal compared to the unwanted interference and noise. Mobile network operators seek to maximize SINR at all sites to deliver the best possible customer experience, either by transmitting at a higher power, or by minimizing the interference and noise. \\

4. CQI calculation - \textcolor{red}{Table published by 3GPP.} \\

Where Pi is the es+mated received power in the pixel from sector i (serving sector) and Pj, Pk, . . . Pu are the es+mated received interference power from all the adjacent sectors. PB is the background noise associated with Sector Si. All the above power values are in Wafs. If the input values are in dBm, they need to be converted to Wafs. The SINR value is a ra+o. If the value needs to be in dB, it must be converted accordingly.

\subsection{Anomaly Detection Entity}

Finally, the \textbf{Anomaly Detection Entity} compares real-time measurements with the results from the Traffic and Coverage Predictors to identify anomalies. If significant deviations between the measured and predicted values are detected, the energy-saving decision-making process is suspended to prevent potential issues.

\textcolor{red}{
Questions To Ask: \\
- Is the decision made periodically? - 1hr prediction, 15 minutes slots (every 15 minutes) \\
- How does this link to the TS xApp? - receive a policy from the rApp \\
- More information is required for the Shutdown part - go through Doc1 again Pulak \\
- What is the output of the algorithm? Does it send a reccomendation to the Near-RT RIC over what to do? - sends a policy, sent as a declarative statement. sent across A1 interface. \\}

\begin{comment}
\begin{algorithm} [H]
\caption{Control Application to the RAN Stack}
\label{alg:control_application}
\SetKwInOut{Input}{Input}
\SetKwInOut{Output}{Output}

\Input{DecisionVariables}

\SetKwProg{Fn}{Function \emph{applyControl}}{}{end}
\Fn{}{
    delay $\gets$ EWMA(delay, delay_in)\;
    
    \While{True}
    {
        \If{$condition1$}
        {
            \Return{$CellBringup$}\;
        }
        \ElseIf{$condition2$}
        {
            \Return{$CellShutdown$}\;
        }
        \Else
        {
            $NoChange$\;
        }
    }
}

\Output{$Control\_Action$} % Adjust the output based on your actual output

\end{algorithm}

\begin{algorithm}[H]
\caption{Cell Shutdown Procedure}
\label{alg:cellshut}
\SetKwProg{generate}{Function \emph{CellShutdown}}{}{end}

Map store=new Map(obj, queue)\;
\generate{Object pivot}{
     \ForAll{child $c$ in pivot}{
     \If{ $c$'s FieldContext is not set and $c$ is fusible}{
          generate($c$)\;
      }
     }
     build pivot's fieldContext $fc$\;
     EmitClassName\;
     EmitFields($fc$)\;
     EmitMethods($fc$)\;
}
\end{algorithm}

\begin{algorithm}
\caption{Cell Shutdown Procedure}
\SetAlgoLined
\KwData{Lat-long of all the transmitters (antenna characteristics, antenna height, etc.)}
\KwData{Field strength at any point from all the transmitters (using CloudRF)}
\KwData{Entire geography subdivided into smaller areas (30m x 30m) each represented by a pixel obtained from CloudRF map}
\For{each pixel}{
    Find the RX power from each BS (more than a certain threshold)\;
    Find the strongest BS\;
    Calculate SINR = $\frac{\text{power of strongest cell}}{\text{sum of all remaining cells' power + noise power}}$\;
}
\For{each BS}{
    Find the list of neighbors\;
    Find the count of pixels where:\;
    \quad I. Serving BS is dominant\;
    \quad II. Neighbor is within a threshold of the serving cell\;
    This count is degree of overlap $C_{ij}$\;
    Total overlap is $C_{ij} + C_{ji}$\;
    Weight of each edge of the undirected graph is the degree of overlap\;
    Calculate the rank $M_i = \frac{K \cdot \sum \text{degree of the node}}{\text{Traffic volume at node (i)}}$ (let $K=1$ for now)\;
}
Sort the list of BS in order of descending rank; the top one is the candidate for shutdown\;
Select the top candidate in rank list above for shutdown\;
With this candidate being shutdown (power = 0), find the SINR distribution (in digital twin)\;
\If{the SINR distribution is within bounds}{
    Break the loop\;
}
\Else{
    Continue to step 1\;
}
\end{algorithm}

\begin{algorithm}
\caption{Cell Bringup Procedure}
\SetAlgoLined
\KwData{For each cell we have desired pattern for CQI distribution for each traffic pattern and time of day}
\For{each cell}{
    Measure the distance between CQI distributions observed and desired\;
    \If{the observed is off by a certain threshold}{
        Turn the cell ON\;
        Recalculate the predicted CQI distribution\;
        \If{the distribution improves}{
            Keep the cell ON\;
        }
        \Else{
            Leave the cell OFF\;
        }
    }
}
\end{algorithm}

\begin{algorithm}
    \SetKwInOut{Input}{Input}
    \SetKwInOut{Output}{Output}

    \underline{function CellBringup} $()$\;
    \Input{Two nonnegative integers $a$ and $b$}
    \Output{$\gcd(a,b)$}
    \eIf{$b=0$}
      {
        return $a$\;
      }
      {
        return Euclid$(b,a\mod b)$\;
      }
    \caption{Algorithm}
\end{algorithm}
\end{comment}

\section{\textcolor{blue}{[WORKSHOP]}Design Rationale}
Methodology.

\subsection{Model Selection}
Model Selection. \textcolor{blue}{Why did we select the model we did?}

\subsection{Data Selection}
Data Selection. \textcolor{blue}{How did we select data to train our model on? How did we know the model will work with less data?}

\subsection{Performance Metrics}
Performance Metrics. \textcolor{blue}{Might not be needed if explained well in previous section.}

\section{Experiments and Results}
CQI + total throughput + energy consumption. \\

- This should otherwise describe the overall system architecture and where does the rApp reside and
with which other components it interacts with.\\
\\
- Setup \\
- Explain why we are taking these results in particular \\
- two/three seperate setups, show CQI and throughput are still good, while energy consumption reduces \\

\section{Conclusions}
Conclusions.

\section{Acknowledgements}
Acknowledgements.

\vspace{12pt}

\end{document}
